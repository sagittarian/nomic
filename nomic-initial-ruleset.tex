% Created 2015-02-10 Tue 12:52
\documentclass[11pt]{article}
\usepackage[utf8]{inputenc}
\usepackage[T1]{fontenc}
\usepackage{fixltx2e}
\usepackage{graphicx}
\usepackage{longtable}
\usepackage{float}
\usepackage{wrapfig}
\usepackage{soul}
\usepackage{textcomp}
\usepackage{marvosym}
\usepackage{wasysym}
\usepackage{latexsym}
\usepackage{amssymb}
\usepackage{hyperref}
\tolerance=1000
\providecommand{\alert}[1]{\textbf{#1}}

\title{nomic-initial-ruleset}
\author{Adam Mesha}
\date{\today}
\hypersetup{
  pdfkeywords={},
  pdfsubject={},
  pdfcreator={Emacs Org-mode version 7.9.3f}}

\begin{document}

\maketitle

\setcounter{tocdepth}{3}
\tableofcontents
\vspace*{1cm}
\section{Nomic - Initial Set of Rules}
\label{sec-1}
\subsection{Immutable Rules}
\label{sec-1-1}

\begin{itemize}
\item 101. All players must always abide by all the rules then in
     effect, in the form in which they are then in effect. The rules
     in the Initial Set are in effect whenever a game begins. The
     Initial Set consists of Rules 101-116 (immutable) and 201-213
     (mutable).
\item 102. Initially rules in the 100's are immutable and rules in the
     200's are mutable. Rules subsequently enacted or transmuted (that
     is, changed from immutable to mutable or vice versa) may be
     immutable or mutable regardless of their numbers, and rules in
     the Initial Set may be transmuted regardless of their numbers.
\item 103. A rule-change is any of the following: (1) the enactment,
     repeal, or amendment of a mutable rule; (2) the enactment,
     repeal, or amendment of an amendment of a mutable rule; or (3)
     the transmutation of an immutable rule into a mutable rule or
     vice versa.

     (Note: This definition implies that, at least initially, all new
     rules are mutable; immutable rules, as long as they are
     immutable, may not be amended or repealed; mutable rules, as long
     as they are mutable, may be amended or repealed; any rule of any
     status may be transmuted; no rule is absolutely immune to
     change.)
\item 104. All rule-changes proposed in the proper way shall be voted
     on. They will be adopted if and only if they receive the required
     number of votes.
\item 105. Every player is an eligible voter. Every eligible voter must
     participate in every vote on rule-changes.
\item 106. All proposed rule-changes shall be written down before they
     are voted on. If they are adopted, they shall guide play in the
     form in which they were voted on.
\item 107. No rule-change may take effect earlier than the moment of
     the completion of the vote that adopted it, even if its wording
     explicitly states otherwise. No rule-change may have retroactive
     application.
\item 108. Each proposed rule-change shall be given a number for
     reference. The numbers shall begin with 301, and each rule-change
     proposed in the proper way shall receive the next successive
     integer, whether or not the proposal is adopted.

     If a rule is repealed and reenacted, it receives the number of
     the proposal to reenact it. If a rule is amended or transmuted,
     it receives the number of the proposal to amend or transmute
     it. If an amendment is amended or repealed, the entire rule of
     which it is a part receives the number of the proposal to amend
     or repeal the amendment.
\item 109. Rule-changes that transmute immutable rules into mutable
     rules may be adopted if and only if the vote is unanimous among
     the eligible voters. Transmutation shall not be implied, but must
     be stated explicitly in a proposal to take effect.
\item 110. In a conflict between a mutable and an immutable rule, the
     immutable rule takes precedence and the mutable rule shall be
     entirely void. For the purposes of this rule a proposal to
     transmute an immutable rule does not ``conflict'' with that
     immutable rule.
\item 111. If a rule-change as proposed is unclear, ambiguous,
     paradoxical, or destructive of play, or if it arguably consists
     of two or more rule-changes compounded or is an amendment that
     makes no difference, or if it is otherwise of questionable value,
     then the other players may suggest amendments or argue against
     the proposal before the vote. A reasonable time must be allowed
     for this debate. The proponent decides the final form in which
     the proposal is to be voted on and, unless the Judge has been
     asked to do so, also decides the time to end debate and vote.
\item 112. The state of affairs that constitutes winning may not be
     altered from achieving n points to any other state of
     affairs. The magnitude of n and the means of earning points may
     be changed, and rules that establish a winner when play cannot
     continue may be enacted and (while they are mutable) be amended
     or repealed.
\item 113. A player always has the option to forfeit the game rather
     than continue to play or incur a game penalty. No penalty worse
     than losing, in the judgment of the player to incur it, may be
     imposed.
\item 114. There must always be at least one mutable rule. The adoption
     of rule-changes must never become completely impermissible.
\item 115. Rule-changes that affect rules needed to allow or apply
     rule-changes are as permissible as other rule-changes. Even
     rule-changes that amend or repeal their own authority are
     permissible. No rule-change or type of move is impermissible
     solely on account of the self-reference or self-application of a
     rule.
\item 116. Whatever is not prohibited or regulated by a rule is
     permitted and unregulated, with the sole exception of changing
     the rules, which is permitted only when a rule or set of rules
     explicitly or implicitly permits it.
\end{itemize}
\subsection{Mutable Rules}
\label{sec-1-2}

\begin{itemize}
\item 201. Players shall alternate in clockwise order, taking one whole
     turn apiece. Turns may not be skipped or passed, and parts of
     turns may not be omitted. All players begin with zero points.

     In mail and computer games, players shall alternate in
     alphabetical order by surname.
\item 202. One turn consists of two parts in this order: (1) proposing
     one rule-change and having it voted on, and (2) throwing one die
     once and adding the number of points on its face to one's score.

     In mail and computer games, instead of throwing a die, players
     subtract 291 from the ordinal number of their proposal and
     multiply the result by the fraction of favorable votes it
     received, rounded to the nearest integer. (This yields a number
     between 0 and 10 for the first player, with the upper limit
     increasing by one each turn; more points are awarded for more
     popular proposals.)
\item 203. A rule-change is adopted if and only if the vote is
     unanimous among the eligible voters. If this rule is not amended
     by the end of the second complete circuit of turns, it
     automatically changes to require only a simple majority.
\item 204. If and when rule-changes can be adopted without unanimity,
     the players who vote against winning proposals shall receive 10
     points each.
\item 205. An adopted rule-change takes full effect at the moment of
     the completion of the vote that adopted it.
\item 206. When a proposed rule-change is defeated, the player who
     proposed it loses 10 points.
\item 207. Each player always has exactly one vote.
\item 208. The winner is the first player to achieve 100 (positive)
     points.

     In mail and computer games, the winner is the first player to
     achieve 200 (positive) points.
\item 209. At no time may there be more than 25 mutable rules.
\item 210. Players may not conspire or consult on the making of future
     rule-changes unless they are team-mates.

     The first paragraph of this rule does not apply to games by mail
     or computer.
\item 211. If two or more mutable rules conflict with one another, or
     if two or more immutable rules conflict with one another, then
     the rule with the lowest ordinal number takes precedence.

     If at least one of the rules in conflict explicitly says of
     itself that it defers to another rule (or type of rule) or takes
     precedence over another rule (or type of rule), then such
     provisions shall supersede the numerical method for determining
     precedence.

     If two or more rules claim to take precedence over one another or
     to defer to one another, then the numerical method again governs.
\item 212. If players disagree about the legality of a move or the
     interpretation or application of a rule, then the player
     preceding the one moving is to be the Judge and decide the
     question. Disagreement for the purposes of this rule may be
     created by the insistence of any player. This process is called
     invoking Judgment.

     When Judgment has been invoked, the next player may not begin his
     or her turn without the consent of a majority of the other
     players.

     The Judge's Judgment may be overruled only by a unanimous vote of
     the other players taken before the next turn is begun. If a
     Judge's Judgment is overruled, then the player preceding the
     Judge in the playing order becomes the new Judge for the
     question, and so on, except that no player is to be Judge during
     his or her own turn or during the turn of a team-mate.

     Unless a Judge is overruled, one Judge settles all questions
     arising from the game until the next turn is begun, including
     questions as to his or her own legitimacy and jurisdiction as
     Judge.

     New Judges are not bound by the decisions of old Judges. New
     Judges may, however, settle only those questions on which the
     players currently disagree and that affect the completion of the
     turn in which Judgment was invoked. All decisions by Judges shall
     be in accordance with all the rules then in effect; but when the
     rules are silent, inconsistent, or unclear on the point at issue,
     then the Judge shall consider game-custom and the spirit of the
     game before applying other standards.
\item 213. If the rules are changed so that further play is impossible,
     or if the legality of a move cannot be determined with finality,
     or if by the Judge's best reasoning, not overruled, a move
     appears equally legal and illegal, then the first player unable
     to complete a turn is the winner.

     This rule takes precedence over every other rule determining the
     winner.
\end{itemize}

\end{document}
